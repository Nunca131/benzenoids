\documentclass{article} 
\begin{document} 

\par\noindent\textbf{\large DAAD SLOVENIA PPP:\\[1em]
  \emph{``''}\\[1.0em]
  Report for Funding Period 2017}

The project started \TODO{when}. 

Focus of the project in the first funding period was the exploration of
graph classes that play a particular role in cheminformatics. Benzenoids
are 2-connected subgraphs of the hexagonal lattice. The visit of the
Leipzig group to Koper in November 2017 was used to initiate the systematic
investigation of convexity in benzenoids. Of particular interest in this
context are the graphs that violate convexity in an extreme manner. Software
was developed for enumerating small and moderate-size benzenoids to screen
for such graphs. 


A joint manuscript on this topic intend for the mathematical chemistry
journal \emph{MATCH} is in perparation.

Therein we introduce a family of benzenoids that
resembles in many aspects to convex benezenoids and thus will be called pseudo-convex
benzenoids. Another family that contains both convex and pseudo-convex benzenoids called 
quasi-convex benzenoids is further introduced. 

A quasi-convex benzenoid is characterized by the fact that the average value of two consecutive 
numbers in its boundary code is never less than $2$. As a generalization of this approach we 
furthermore define a topological index for benzenoids that is called the convexity number. Finally, 
we investigate convexity numbers of several important families of benzenoids.

Current activities include in particular the construction of publicly
available web service to explore the universe of benzenoid graphs. 

\end{document}
